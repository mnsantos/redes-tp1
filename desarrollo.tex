Implementamos un programa para realizar la captura de paquetes ARP que además parsea y obtiene resultados a partir de la informacion recopilada. La captura de los paquetes se llevó a cabo con la ayuda de la librería Scapy de Python.

Ejecutamos el programa en distintas LANs con distintas caracteristicas: redes hogare\~nas controladas y redes públicas no controladas. Luego analizamos y comparamos los resultados de cada una.\

En la red doméstica se usaron 4 dispositivos conectados mediante interfaz wifi y 3 mediante interfaz ethernet.

Las redes públicas fueron dos Starbucks, en los cuales había alrededor de 20 o 30 personas y veíamos gente usando computadora o celular, posiblemente conectada a la red local.

En el caso de la red doméstica, hicimos dos análisis distintos, uno con y otro sin intervención. Con esto nos referimos a que en un primer análisis dejamos los dispositivos en idle y el programa corriendo cerca de 2 horas mientras
que en el segundo, conectamos y desconectamos los distintos dispositivos a la red cada 10 minutos en el mismo periodo de tiempo. Lo que esperamos es que se generen mas paquetes ARP al desconectar y conectar los dispositivos que al dejarlos en idle.

El análisis se hizo sobre dos fuentes de información:

\begin{itemize}
\item $S_{dst}$ = \{ $s_{1},s_{2} \ldots s_{n} $ \}, siendo $s_{i}$ una IP que aparece como dirección destino en los paquetes
ARP \emph{who-has}
\item $S_{src}$ = \{ $s_{1},s_{2} \ldots s_{n} $ \}, siendo $s_{i}$ una IP que aparece como dirección origen en los paquetes
ARP \emph{who-has}
\end{itemize}

El objetivo de analizar estas fuentes de información es encontrar nodos distinguidos en la red. Como resultado esperamos ver que independientemente de la red, sea el router (gateway) el nodo que aparezca mayor cantidad de veces como destinatario y 
emisor de mensajes ARP \emph{who-has}.
En otras palabras, que para ambas fuentes de informacion, la IP del router sea el símbolo con mayor probabilidad en Sdst y Ssrc y, por lo tanto, con menor cantidad de información por paquete enviado.

Para el análisis volcamos los datos recopilados en una serie de gráficos que nos permitieron un mejor entendimiento del comportamiento de cada red. Además calculamos la entropía de cada una de las fuentes de información y la comparamos con la cantidad de información de cada nodo.

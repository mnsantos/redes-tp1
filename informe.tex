\documentclass[a4paper,11pt]{article}
\usepackage[utf8]{inputenc}
\usepackage[paper=a4paper, hmargin=1.5cm, bottom=1.5cm, top=3.5cm]{geometry}
\usepackage[T1]{fontenc}
\usepackage[spanish]{babel}
\usepackage[colorlinks=true, linkcolor=blue]{hyperref} %Links para el indice.
\usepackage{amsfonts}
\usepackage{verbatim}
\usepackage{listings}
\usepackage{caption}
\usepackage{subcaption}
\usepackage{graphicx}

\usepackage[section]{placeins}
\usepackage{float}
%\usepackage{adjustbox}
\usepackage{amsmath}
\usepackage{blindtext}
\usepackage{sidecap}
\usepackage{color}

% \newcommand{\real}{\hbox{\bf R}}

\title{Trabajo Práctico de Teoría de las Comunicaciones: Análisis de protocolo \emph{ARP} mediante la escucha de mensajes \emph{who-has} }

\begin{document}

\maketitle

\begin{center}
	Universidad de Buenos Aires - Departamento de Computaci\'on - FCEN
\end{center}

\vspace{2cm}
Integrantes:

\begin{itemize}
	\item Matayoshi, Leandro L.U.: 79/11 \verb+leandro.matayoshi@gmail.com+
	\item Melnik, Jonathan L.U.: 571/09 \verb+jonathanmelnik@gmail.com+
	\item Santos, Martín L.U.: 413/11 \verb+martin.n.santos@gmail.com+
		
\end{itemize}

\newpage

\tableofcontents

\newpage

\section{Introducción}

\section{Desarrollo}

Implementamos un programa para realizar la captura de paquetes ARP que además parsea y obtiene resultados a partir de la informacion recopilada. La captura de los paquetes se llevó a cabo con la ayuda de la librería Scapy de Python.

Ejecutamos el programa en distintas LANs con distintas caracteristicas: redes hogare\~nas controladas y redes públicas no controladas. Luego analizamos y comparamos los resultados de cada una.\

En la red doméstica se usaron 4 dispositivos conectados mediante interfaz wifi y 3 mediante interfaz ethernet.

Las redes públicas fueron dos Starbucks y un centro médico, en los cuales había alrededor de 20 o 30 personas y veíamos gente usando computadora o celular, posiblemente conectada a la red local.

En el caso de la red doméstica, hicimos dos análisis distintos, uno con y otro sin intervención. Con esto nos referimos a que en un primer análisis dejamos los dispositivos en idle y el programa corriendo cerca de 2 horas mientras
que en el segundo, conectamos y desconectamos los distintos dispositivos a la red cada 10 minutos en el mismo periodo de tiempo. Lo que esperamos es que se generen mas paquetes ARP al desconectar y conectar los dispositivos que al dejarlos en idle.

El análisis se hizo sobre dos fuentes de información:

\begin{itemize}
\item $S_{dst}$ = \{ $s_{1},s_{2} \ldots s_{n} $ \}, siendo $s_{i}$ una IP que aparece como dirección destino en los paquetes
ARP \emph{who-has}
\item $S_{src}$ = \{ $s_{1},s_{2} \ldots s_{n} $ \}, siendo $s_{i}$ una IP que aparece como dirección origen en los paquetes
ARP \emph{who-has}
\end{itemize}

El objetivo de analizar estas fuentes de información es encontrar nodos distinguidos en la red. Como resultado esperamos ver que independientemente de la red, sea el router (gateway) el nodo que aparezca mayor cantidad de veces como destinatario y 
emisor de mensajes ARP \emph{who-has}.
En otras palabras, que para ambas fuentes de informacion, la IP del router sea el símbolo con mayor probabilidad en Sdst y Ssrc y, por lo tanto, con menor cantidad de información por paquete enviado.

Para el análisis volcamos los datos recopilados en una serie de gráficos que nos permitieron un mejor entendimiento del comportamiento de cada red. Además calculamos la entropía de cada una de las fuentes de información y la comparamos con la cantidad de información de cada nodo.


\section{Resultados y análisis}

A continuación explicaremos los resultados obtenidos en cada una de las redes.

Como mencionamos, en la red doméstica realizamos la recopilación de los datos con y sin intervención. Pudimos ver que al desconectar y conectar los dispositivos a la red, se generaban alrededor de 20 a 30 paquetes ARP. De modo que encontramos diferencias entre ambas experimentaciones. Los gráficos siguientes muestran un aspecto de ambas corridas: cada nodo del grafo es un nodo de la red, y los ejes dirigidos apuntan de un nodo v a otro w si hubo algun mensaje tal que el pSrc es igual al IP de v y el pDst es igual al IP de w. Además los nodos varían en tama\~no de acuerdo a la cantidad de paquetes a ARP que emitieron, siendo los más grandes los nodos que emitieron más mensajes ARP:

\begin{figure}[!h]
	\begin{center}
		  \includegraphics[scale=0.4]{Graficos/grafico_1.png}
		  \caption{Red doméstica, experimento 1}
		  \label{fig:contra1}
	\end{center}
\end{figure}

\begin{figure}[!h]
	\begin{center}
		  \includegraphics[scale=0.4]{Graficos/grafico_2.png}
		  \caption{Red doméstica, experimento 2}
		  \label{fig:contra1}
	\end{center}
\end{figure}

En ambos gráficos se ve claramente cuál es el gateway de la red, en este caso 192.168.0.1, ya que todos los nodos se comunican con ese nodo. Además es el nodo que envía mas mensajes ARP, lo cual se ve por su tama\~no mayor a los demás. 

En la segunda experimentación, a pesar de que el gateway continúa siendo el nodo de mayor tama\~no, hay otros que son casi iguales a él, lo que indica que enviaron casi la misma cantidad de paquetes ARP. Analizando los paquetes enviados por esos nodos, vimos que muchos son enviados al router, pero que gran parte de esos paquetes son enviados a la direccion 169.254.255.255. Segun la documentacion de Apple[1](los dispositivos son en su mayoria de Apple) esos paquetes son causados porque los dispositivos no encuentran el servidor DHCP, esto sucede cuando se los desconecta ya que pierden la ruta al servidor y entonces cuando se los vuelve a conectar, entonces envían mensajes ARP a esa direccion. 

Tambien vimos que hay paquetes que son enviados desde y hacia la direccion 181.167.64.234, la cual es la dirección pública que tiene el router a internet. Vemos que intercambia mensajes con  181.167.64.1 y creemos que es un router del ISP. Hay otros mensajes enviados desde esa dirección hacia la red local, pero desconocemos la razón de estos mensajes ARP. 

Tampoco sabemos la procedencia de otras IPs que parecieran ser dispositivos de la red(192.168.0.x) pero que no pudimos identificar y hay otras IPs que parecieran estar fuera de la red como 10.119.163.64 y 10.119.128.1, pero aún así están enviando mensajes ARP con broadcast en la red local. Realizando una búsqueda en internet no pudimos encontrar información sobre esas IPs.

Realizamos otra experimentación en una red pública(Starbucks). Dado que la red es pública, no teníamos conocimiento alguno de los dispositivos en ella y estos se desconectaban y conectaban aleatoriamente. Obtuvimos los siguientes resultados:

\begin{figure}[!h]
	\begin{center}
		  \includegraphics[scale=0.4]{Graficos/grafico_3.png}
		  \caption{Red de Starbucks}
		  \label{fig:contra1}
	\end{center}
\end{figure}

Nuevamente es sencillo identificar cuál es el gateway en la red: 10.251.14.1. Lo que nos llamó mucho la atención es la existencia de dos nodos(los más grandes) que emitían paquetes ARP casi de forma constante. El mayor de ellos mandaba muchos de sus paquetes a la direccion 0.0.0.0. Suponemos que ese funcionamiento es particular de la red de Starbucks que usa FibertelZone como una capa intermedia para conectarse.



\section{Conclusiones}

\section{Referencias}

[1] http://support.apple.com/kb/ta2600

\end{document}

\emph{Adress Resolution Protocol} o \emph{ARP} es un protocolo de la capa de enlace de datos responsable de encontrar
la dirección hardware (Ethernet \emph{MAC}) que corresponde a una determinada dirección IP. Los nodos que se comunican
a través de este protocolo deben formar parte de una misma red (no permite comunicación entre distintas redes).

Uno de los aspectos del protocolo que nos interesa analizar 
se basa en mensajes broadcast (a los cuales llamamos \emph{who-has}) utilizados por los nodos de la red
para obtener la dirección física del router o gateway (punto de conexión entre la red local y las demás redes).
Las respuestas por parte del router, por el contrario, son enviadas a través de mensajes \emph{unicast} ( 
también conocidos como \emph{is-at}).

Para que el protocolo sea llevado a cabo con éxito, en cada mensaje debe estar incluída la información necesaria:

\begin{itemize}
\item Tipo de mensaje: \emph{who-has} o \emph{is-at} en el contexto de nuestro análisis. (Existen otros tipos de mensajes)
\item Especificaciones de protocolo 
\item \emph{MAC} de la fuente
\item \emph{MAC} del destino 
\item \emph{IP} de la fuente
\item \emph{IP} de la fuente
\end{itemize}

A continuación incluímos como ejemplo 2 mensajes capturados por el script de python basado en \emph{scapy}, el primero del tipo 
\emph{who-has} emitido por un nodo de una red (computadora) y el segundo del tipo \emph{is-at}, emitido por el router
de la misma:

\vspace{1cm}
Paquete \emph{\textbf{who-has}}. Consulta de un nodo al router:

\begin{itemize}
\item src = 0c:77:1a:ed:a7:42" (campo agregado por scappy)
\item dst = ff:ff:ff:ff:ff:ff (campo agregado por scappy)
\item hwdst = 00:00:00:00:00:00 (MAC destino = 0 en el caso de no conocerla)
\item hwsrc = 0c:77:1a:ed:a7:42 (MAC fuente)
\item psrc = 10.251.14.60 (IP fuente)
\item pdst = 10.251.14.1 (IP destino)
\item op = 1 (Tipo de mensaje \emph{who-has})
\end{itemize}

\vspace{1cm}
Paquete \emph{\textbf{is-at}}. Respuesta del router a la máquina desde donde corrimos el script (distinto del nodo emisor del
mensaje anterior)

\begin{itemize}
\item src = 4c:4e:35:12:a5:9c (campo agregado por scappy)
\item dst = 6c:71:d9:56:56:b9 (campo agregado por scappy)
\item hwdst = 6c:71:d9:56:56:b9 (MAC destino)
\item hwsrc = 4c:4e:35:12:a5:9c (MAC fuente)
\item psrc = 10.251.14.1 (IP fuente)
\item pdst = 10.251.14.90 (IP destino)
\item op = 2 (Tipo de mensaje \emph{is-at})
\end{itemize}


